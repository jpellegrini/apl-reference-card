% Reference Card for APL
%**start of header
\newcount\columnsperpage

% The format for this file is adapted from the GNU Emacs reference
% card version 1.9, by Stephen Gildea.

% This file can be printed with 1, 2, or 3 columns per page (see below).
% Specify how many you want here.

\columnsperpage=3

% PDF output layout.  0 for A4, 1 for letter (US), a `l' is added for
% a landscape layout.

\input pdflayout.sty
\pdflayout=(1l)

% Nothing else needs to be changed.
% Typical command to format:  tex calccard.tex
% Typical command to print (3 cols):  dvips -t landscape calccard.dvi

% Copyright (C) 2020 Jeronimo Pellegrini, Lucas S. Vieira

% This document is free software: you can redistribute it and/or modify
% it under the terms of the GNU General Public License as published by
% the Free Software Foundation, either version 3 of the License, or
% (at your option) any later version.

% As a special additional permission, you may distribute reference cards
% printed, or formatted for printing, with the notice "Released under
% the terms of the GNU General Public License version 3 or later"
% instead of the usual distributed-under-the-GNU-GPL notice, and without
% a copy of the GPL itself.

% This document is distributed in the hope that it will be useful,
% but WITHOUT ANY WARRANTY; without even the implied warranty of
% MERCHANTABILITY or FITNESS FOR A PARTICULAR PURPOSE.  See the
% GNU General Public License for more details.

% You should have received a copy of the GNU General Public License
% along with GNU APL.  If not, see <https://www.gnu.org/licenses/>.

% This file is intended to be processed by plain TeX (TeX82).
%
% The final reference card has six columns, three on each side.
% This file can be used to produce it in any of three ways:
% 1 column per page
%    produces six separate pages, each of which needs to be reduced to 80%.
%    This gives the best resolution.
% 2 columns per page
%    produces three already-reduced pages.
%    You will still need to cut and paste.
% 3 columns per page
%    produces two pages which must be printed sideways to make a
%    ready-to-use 8.5 x 11 inch reference card.
%    For this you need a dvi device driver that can print sideways.
% Which mode to use is controlled by setting \columnsperpage above.
%
% Authors (Calc reference card):
%  Jeronimo Pellegrini <j_p@aleph0.info>
%  Lucas S. Vieira <lucasvieira@protonmail.com>
%
% Author (refcard.tex format):
%  Stephen Gildea <stepheng+emacs@gildea.com>

\input version.tex

\def\shortcopyrightnotice{\vskip 1ex plus 2 fill
  \centerline{\small \copyright\ \year\ Jeronimo Pellegrini, Lucas S. Vieira
  Permissions on back.}}

\def\copyrightnotice{
\vskip 1ex plus 2 fill\begingroup\small
\centerline{Copyright \copyright\ \year\ Jeronimo Pellegrini, Lucas S. Vieira}
\centerline{designed by Jeronimo Pellegrini, Lucas S. Vieira and Stephen Gildea,}
\centerline{for GNU APL}

Released under the terms of the GNU General Public License version 3 or later.
For more APL documentation, check the GNU APL site.
For the \TeX{} source for this card,
see:\par
{\small https://www.github.com/jpellegrini/gnu-apl-refcard}
\endgroup}

% make \bye not \outer so that the \def\bye in the \else clause below
% can be scanned without complaint.
\def\bye{\par\vfill\supereject\end}

\newdimen\intercolumnskip
\newbox\columna
\newbox\columnb

\def\ncolumns{\the\columnsperpage}

\message{[\ncolumns\space
  column\if 1\ncolumns\else s\fi\space per page]}

\def\scaledmag#1{ scaled \magstep #1}

% This multi-way format was designed by Stephen Gildea
% October 1986.
\if 1\ncolumns
  \hsize 4in
  \vsize 10in
  \voffset -.7in
  \font\titlefont=\fontname\tenbf \scaledmag3
  \font\headingfont=\fontname\tenbf \scaledmag2
  \font\smallfont=\fontname\sevenrm
  \font\smallsy=\fontname\sevensy

  \footline{\hss\folio}
  \def\makefootline{\baselineskip10pt\hsize6.5in\line{\the\footline}}
\else
  \hsize 3.2in
  \vsize 7.95in
  \hoffset -.75in
  \voffset -.745in
  \font\titlefont=cmbx10 \scaledmag2
  \font\headingfont=cmbx10 \scaledmag1
  \font\smallfont=cmr6
  \font\smallsy=cmsy6
  \font\eightrm=cmr8
  \font\eightbf=cmbx8
  \font\eightit=cmti8
  \font\eighttt=cmtt8
  \font\eightsy=cmsy8
  \textfont0=\eightrm
  \textfont2=\eightsy
  \def\rm{\eightrm}
  \def\bf{\eightbf}
  \def\it{\eightit}
  \def\tt{\eighttt}
  \normalbaselineskip=.8\normalbaselineskip
  \normallineskip=.8\normallineskip
  \normallineskiplimit=.8\normallineskiplimit
  \normalbaselines\rm		%make definitions take effect

  \if 2\ncolumns
    \let\maxcolumn=b
    \footline{\hss\rm\folio\hss}
    \def\makefootline{\vskip 2in \hsize=6.86in\line{\the\footline}}
  \else \if 3\ncolumns
    \let\maxcolumn=c
    \nopagenumbers
  \else
    \errhelp{You must set \columnsperpage equal to 1, 2, or 3.}
    \errmessage{Illegal number of columns per page}
  \fi\fi

  \intercolumnskip=.46in
  \def\abc{a}
  \output={%
      % This next line is useful when designing the layout.
      %\immediate\write16{Column \folio\abc\space starts with \firstmark}
      \if \maxcolumn\abc \multicolumnformat \global\def\abc{a}
      \else\if a\abc
	\global\setbox\columna\columnbox \global\def\abc{b}
        %% in case we never use \columnb (two-column mode)
        \global\setbox\columnb\hbox to -\intercolumnskip{}
      \else
	\global\setbox\columnb\columnbox \global\def\abc{c}\fi\fi}
  \def\multicolumnformat{\shipout\vbox{\makeheadline
      \hbox{\box\columna\hskip\intercolumnskip
        \box\columnb\hskip\intercolumnskip\columnbox}
      \makefootline}\advancepageno}
  \def\columnbox{\leftline{\pagebody}}

  \def\bye{\par\vfill\supereject
    \if a\abc \else\null\vfill\eject\fi
    \if a\abc \else\null\vfill\eject\fi
    \end}
\fi

% we won't be using math mode much, so redefine some of the characters
% we might want to talk about
\catcode`\^=12
\catcode`\_=12

\chardef\\=`\\
\chardef\{=`\{
\chardef\}=`\}

\hyphenation{mini-buf-fer}

\parindent 0pt
\parskip 1ex plus .5ex minus .5ex

\def\small{\smallfont\textfont2=\smallsy\baselineskip=.8\baselineskip}

\outer\def\newcolumn{\vfill\eject}

\outer\def\title#1{{\titlefont\centerline{#1}}\vskip 1ex plus .5ex}

%% removed filbreak from definition of section, in order to make the
%% content fit...
%% -- Jeronimo
\outer\def\section#1{\par%\filbreak
  \vskip 3ex plus 2ex minus 2ex {\headingfont #1}\mark{#1}%
  \vskip 2ex plus 1ex minus 1.5ex}

\newdimen\keyindent

\def\beginindentedkeys{\keyindent=1em}
\def\endindentedkeys{\keyindent=0em}
\endindentedkeys

\def\paralign{\vskip\parskip\halign}

\def\<#1>{$\langle${\rm #1}$\rangle$}

\def\kbd#1{{\tt#1}\null}	%\null so not an abbrev even if period follows

\def\beginexample{\par\leavevmode\begingroup
  \obeylines\obeyspaces\parskip0pt\tt}
{\obeyspaces\global\let =\ }
\def\endexample{\endgroup}

\def\key#1#2{\leavevmode\hbox to \hsize{\vtop
  {\hsize=.75\hsize\rightskip=1em
  \hskip\keyindent\relax#1}\kbd{#2}\hfil}}

% akey has the right size to the left, made specially for the COMMANDS
% sections, so no lines would overfill
\def\akey#1#2{\leavevmode\hbox to \hsize{\vtop
  {\hsize=.64\hsize\rightskip=1em
  \hskip\keyindent\relax#1}\kbd{#2}\hfil}}

% akey has the right size to the left, made specially for the NOTATION
% section , so no lines would overfill
\def\bkey#1#2{\leavevmode\hbox to \hsize{\vtop
  {\hsize=.65\hsize\rightskip=1em
  \hskip\keyindent\relax#1}\kbd{#2}\hfil}}
  
\def\ckey#1#2{\leavevmode\hbox to \hsize{\vtop
  {\hsize=.65\hsize\rightskip=1em
  \hskip\keyindent\relax#1}{\tt {#2}}\null}}



\newbox\metaxbox
\setbox\metaxbox\hbox{\kbd{M-x }}
\newdimen\metaxwidth
\metaxwidth=\wd\metaxbox

\def\metax#1#2{\leavevmode\hbox to \hsize{\hbox to .75\hsize
  {\hskip\keyindent\relax#1\hfil}%
  \hskip -\metaxwidth minus 1fil
  \kbd{#2}\hfil}}

\def\threecol#1#2#3{\hskip\keyindent\relax#1\hfil&\kbd{#2}\quad
  &\kbd{#3}\quad\cr}

%
% Calc-specific commands here:
%

\let\^=^
\let\_=_
\catcode`\^=7
\catcode`\_=8

% Redefine to make spaces a bit smaller
\let\wkbd=\kbd
\def\kbd#1{{\spaceskip=.37em\tt#1}\null}

\def\wkey#1#2{\leavevmode\hbox to \hsize{\vtop
  {\hsize=.75\hsize\rightskip=1em
  \hskip\keyindent\relax#1}\wkbd{#2}\hfil}}
\def\wthreecol#1#2#3{\hskip\keyindent\relax#1\hfil&\wkbd{#2}\quad
  &\wkbd{#3}\quad\cr}

\def\stkkey#1#2#3#4{\par\line{\hskip1em\rlap{\kbd{#1}}\hskip4.5em%
  \rlap{{#2}}\hskip7.5em\rlap{{#3}}\hskip7.5em\rlap{{#4}}\hfill}\par}
\def\S#1{$S_{\scriptscriptstyle #1}$}
\def\swap{$\leftrightarrow$}

\def\calcprefix{C-x *\ }
\def\,{{\rm ,\hskip.55em}\ignorespaces}
\def\lesssectionskip{\vskip-1.5ex}

\def\iline#1{\par\line{\hskip1em\relax #1\hfill}\par}

\if 1\ncolumns
\else
  \font\eighti=cmmi8
  \textfont1=\eighti
\fi

%**end of header


% Column 1

\input wasyfont

\title{GNU APL Reference Card}

\centerline{(for GNU APL version \aplversion)}

\def\rul{$\underline{\phantom{AAAAAAAAAAAAAAAAAAAAAA}}$}

% break line, it's too close to the header:
\hfill\break

Use {\tt C-x gnu-apl} to start GNU APL in Emacs.

\section{Emacs mode}
{\bf Interaction mode:}\par
\key{beginning of defun}{C-M-a}
\key{end of defun}{C-M-e}
\key{find function at point}{M-.}
\key{apropos symbol}{C-c C-a}
\key{edit function}{C-c C-f}
\key{show help for symbol}{C-c C-h}
\key{finnapl list}{C-c TAB}
\key{show keyboard}{C-c C-k}
\key{plot line}{C-c RET}
\key{edit variable}{C-c C-v}
\key{trace}{C-c C-.}

{\bf Edit mode:}\par
% C-c             Prefix Command
% ESC             Prefix Command
% insert space    s-SPC           gnu-apl-insert-spc
\key{go to beginning of defun}{C-M-a}
\key{go to end of defun}{C-M-e}
\key{find function at point}{M-.}
\key{apropos symbol}{C-c C-a}
\key{interactive send current function}{C-c C-c}
\key{help for symbol}{C-c C-h}
\key{finnapl list}{C-c TAB}
\key{show keyboard}{C-c C-k}
\key{interactive send buffer}{C-c C-l}
\key{interactive send region}{C-c C-s}
\key{switch to interactive}{C-c C-z}
\key{trace}{C-c C-.}
\key{indent}{C-M-q}
\section{System}
Notation for commands:
% typesetting with tabs: TeX Book, p.232
\settabs\+\indent&XX &xxxxxxxxxxxxxxxxxxx&XX&xxxxxxxxxxxxxxxxxxx&XX&xxxxxxxxxxxxxxxxxxx\cr
\+&&&&&&\cr
\+&{\tt F}& filename         & {\tt L}& library   & {\tt P}& path    \cr
\+&{\tt G}& logging facility & {\tt O}& object    & {\tt S}& symbol  \cr
\+&{\tt W}& workspace        &        &           &        &         \cr
{\bf APL standard commands}\par
\akey{check workspace intergity}{)CHECK}
\akey{clear workspace}{)CLEAR}
\akey{save workspace as {\tt CONTINUE} and exit}{)CONTINUE}
\akey{copies objects from given workspace}{)COPY [L] W [O $\ldots$]}
\akey{remove W}{)DROP [L] W}
\akey{dump W (readable, HTML escaped)}{)DUMP-HTML [[L] W]}
\akey{dump W (readable APL)}{)DUMP [[L] W]}
\akey{dump W (readable APL, verbose)}{)DUMPV [[L] W]}
\akey{erase symbol(s)}{)ERASE S $\ldots$}
\akey{show functions}{)FNS [from-to]}
\akey{help}{)HELP [primitive]}
\akey{history}{)HIST [CLEAR]}
\akey{runs command on host}{)HOST command $\ldots$}
\akey{loads workspace (IBM .atf format)}{)IN F [O $\ldots$]}
\akey{show libraries and paths}{)LIBS [[L] path]}
\akey{show saved workspaces}{)LIB [L|P] [from-to]}
\akey{load workspace W}{)LOAD [L] W}
\akey{show more error info}{)MORE}
\akey{lists symbols matching name}{)NMS [from-to]}
\akey{quit APL}{)OFF}
\akey{show operators}{)OPS [from-to]}
\akey{dump workspace (IBM .atf format)}{)OUT name [O $\ldots$]}
\akey{protects during copying}{)PCOPY [L] W [O $\ldots$]}
\akey{protects during loading}{)PIN F [O $\ldots$]}
\akey{quiet load}{)QLOAD [[L] W]}
\akey{reset state indicator}{)RESET }
\akey{save workspace as W}{)SAVE [[L] W]}
\akey{clear suspended functions}{)SIC}
\akey{see suspended functions and locals}{)SINL}
\akey{see suspended functions}{)SIS}
\akey{state indicator}{)SI}
\akey{show symbol count}{)SYMBOLS [count]}
\akey{show values in use by interpreter}{)VALUES}
\akey{show variables}{)VARS [from-to]}
\akey{get/set workspace ID}{)WSID [W]}

{\bf GNU extension commands (mostly for debugging)}\par
\akey{toggles boxing of values when printing}{]BOXING [OFF|num]}
\akey{toggle colored output}{]COLOR [ON|OFF]}
\akey{dump W in HTML file}{]DOXY [path]}
\akey{expected error count in test suite}{]EXPECT error\_count}
\akey{help}{]HELP [primitive]}
\akey{show keyboard layout}{]KEYB}
\akey{as )LIB, but shows fil eextensions}{]LIB [L|P] [from-to]}
\akey{show/set logging facilities}{]LOG [G [ON|OFF]]}
\akey{next testcase file}{]NEXTFILE}
\akey{performance statistics}{]PSTAT [CLEAR|SAVE]}
\akey{as )SIS, with more details}{]SIS}
\akey{as )SI, with more details}{]SI}
\akey{shared variables}{]SVARS}
\akey{describe internal details of symbol S}{]SYMBOL S}
\akey{define user command}{]USERCMD [ $\ldots$ ]}
\akey{toggle output coloring on console}{]XTERM [ON|OFF]}

{\bf System variables:}\par
\wkey{character input/output}{$\APLinput$}
\wkey{evaluated input/output}{$\APLbox$}
\wkey{account information}{$\APLbox$AI}
\wkey{command line arguments}{$\APLbox$ARG}
\wkey{atomic vector}{$\APLbox$AV}
\wkey{comparison tolerance}{$\APLbox$CT}
\wkey{event message}{$\APLbox$EM}
\wkey{event type}{$\APLbox$ET}
\wkey{format control}{$\APLbox$FC}
\wkey{index origin (indexes start: 1, can be set to 0)}{$\APLbox$IO}
\wkey{left argument}{$\APLbox$L}
\wkey{line counters}{$\APLbox$LC}
\wkey{latent expression (executed when workspace is loaded)}{$\APLbox$LX}
\wkey{print precision (number of digits)}{$\APLbox$PP}
\wkey{print style}{$\APLbox$PS}
\wkey{print width (max characters in each printed line)}{$\APLbox$PW}
\wkey{right argument}{$\APLbox$R}
\wkey{random link}{$\APLbox$RL}
\wkey{shared variable event}{$\APLbox$SVE}
\wkey{system limits}{$\APLbox$SYL}
\wkey{terminal control characters}{$\APLbox$TC}
\wkey{time stamp (current time)}{$\APLbox$TS}
\wkey{time zone (offset from GMT)}{$\APLbox$TZ}
\wkey{user load}{$\APLbox$UL}
\wkey{axis argument}{$\APLbox$X}
\wkey{workspace available (bytes for workspace)}{$\APLbox$WA}
\wkey{dfn axis argument}{X}
\wkey{dfn result}{$\lambda$}
\wkey{dfn left value arg}{$\alpha$}
\wkey{dfn left function arg}{$\underline{\alpha}$}
\wkey{dfn right value arg}{$\omega$}
\wkey{dfn right function arg}{$\underline{\omega}$}

{\bf System functions:}\par
\wkey{atomic function}{$\APLbox$AF}
\wkey{attributes}{$\APLbox$AT}
\wkey{char representation}{$\APLbox$CR}
\wkey{delay}{$\APLbox$DL}
\wkey{D. Knuth's dancing links}{$\APLbox$DLX}
\wkey{execute alternate}{$\APLbox$EA}
\wkey{execute both}{$\APLbox$EB}
\wkey{execute controlled}{$\APLbox$EC}
\wkey{environment}{$\APLbox$ENV}
\wkey{event simulate}{$\APLbox$ES}
\wkey{expunge}{$\APLbox$EX}
\wkey{fast Fourier transform}{$\APLbox$FFT}
\wkey{file I/O}{$\APLbox$FIO}
\wkey{FiX (FFI/call native functions)}{$\APLbox$FX}
\wkey{Gtk GUI}{$\APLbox$GTK}
\wkey{MAP ravel elements}{$\APLbox$MAP}
\wkey{input from script}{$\APLbox$INP}
\wkey{name association}{$\APLbox$NA}
\wkey{name class}{$\APLbox$NC}
\wkey{name list}{$\APLbox$NL}
\wkey{plot a graph}{$\APLbox$PLOT}
\wkey{regular expression, regex $\APLbox$RE string}{$\APLbox$RE}
\wkey{random APL value}{$\APLbox$RVAL}
\wkey{state indicator}{$\APLbox$SI}
\wkey{SQL functions}{$\APLbox$SQL}
\wkey{shared variable control}{$\APLbox$SVC}
\wkey{shared variable offer}{$\APLbox$SVO}
\wkey{shared variable query}{$\APLbox$SVQ}
\wkey{shared variable retraction}{$\APLbox$SVR}
\wkey{shared variable state}{$\APLbox$SVS}
\wkey{STOP vector}{$\APLbox$STOP}
\wkey{transfer form}{$\APLbox$TF}
\wkey{TRACE vector}{$\APLbox$TRACE}
\wkey{unicode character}{$\APLbox$UCS}
\section{Notation}
\bkey{comment}{$\APLcomment$}
\bkey{statement separator}{$\diamond$}
\bkey{assignment}{A$\leftarrow\;\ldots$}
\bkey{assignment}{(A B C)$\leftarrow\;\ldots\;\ldots\;\ldots$}
\bkey{function definition}{$\APLdown$}
\bkey{\rul}{}
\bkey{zilde (empty vector)}{$\APLnot{O}$}
\bkey{a}{+ a}
\bkey{a + b}{a + b}
\bkey{- a}{- a}
\bkey{a - b}{a - b}
\bkey{magnitude of a}{| a}
\bkey{b mod a}{a | b}
\bkey{signal (-1, 0, +1)}{$\times$ a}
\bkey{ab}{a $\times$ b}
\bkey{$1/$a}{$\div$ a}
\bkey{a$/$b}{a $\div$ b}
\bkey{floor of a}{$\lfloor$a}
\bkey{min(a,b)}{a$\lfloor$b}
\bkey{ceiling of a}{$\lceil$a}
\bkey{max(a,b)}{a$\lceil$b}
\bkey{$e^{\tt a}$}{* a}
\bkey{${\tt a}^{\tt b}$}{a * b}
\bkey{$\log$(a)}{$\APLlog$ a}
\bkey{$\log_{\hbox{b}}$(a)}{b $\APLlog$ a}
\bkey{first $n$ non-negative integers}{$\iota$ n}
\bkey{\rul}{}
\bkey{a = b}{a = b}     
\bkey{a $<$ b}{a < b}     
\bkey{a $>$ b}{a > b}     
\bkey{a $\leq$ b}{a $\leq$ b}
\bkey{a $\geq$ b}{a $\geq$ b}
\bkey{expression max depth}{$\equiv$ a}
\bkey{match (value and type)}{a $\equiv$ b}
\bkey{expression min depth}{$\not\equiv$ a}
\bkey{not match}{a $\not\equiv$ b}
\bkey{not a}{$\APLnot*$ a}
\bkey{a or b}{a $\vee$ b}
\bkey{a and b}{a $\wedge$ b}
\bkey{a nor b}{a $\APLnot{\lor}$ b}
\bkey{a nand b}{a $\APLnot{\land}$ b}
\bkey{a $\in$ b ?}{a $\in$ b}
\bkey{find a in b (binary index)}{a $\underline{\in}$ b ?}
\bkey{\rul}{}
\bkey{a!}{!a}
\bkey{${b \choose a}$}{a!b}
\bkey{\rul}{}
\bkey{a$\pi$}{$\APLcirc*$a}
\bkey{circle (trig) function}{a $\APLcirc*$ b}
\bkey{random integer in [1,a]}{?a}
\bkey{a distinct random integers in [1,b]}{a?b}
\bkey{\rul}{}
\bkey{makes a vector out of A}{, A}
\bkey{append B to A}{A,B}
\bkey{number of components in each dimension of A}{$\rho$ A}
\bkey{array with shape A and data elements B}{A$\rho$ B}
\bkey{inverse matrix of A}{$\APLinv$ A}
\bkey{B$^{-1}$A (solution to $Bx=A$)}{A$\APLinv$B}
\bkey{reverse elements of A (1$^{st}$ index)}{$\ominus$A}
\bkey{rotate B by A positions}{A$\ominus$B}
\bkey{reverse elements of A (last index)}{$\APLvert{\Circle}$A}
\bkey{rotate B by A positions (last index)}{A$\APLvert{\Circle}$B}
\bkey{drop first A elements of B}{A$\downarrow$B}
\bkey{select first A elements of B}{A$\uparrow$B}
\bkey{intersection}{A$\cap$B}
\bkey{set (remove duplicates)}{$\cup$A}
\bkey{union}{A$\cup$B}
\bkey{identity}{$\vdash$A}
\bkey{take right hand side (B)}{A$\vdash$B}
\bkey{null}{$\dashv$A}
\bkey{take left hand side (A)}{A$\dashv$B}
\bkey{i-th element of A}{A[i]}
\bkey{elements of A with indices i, j, k, $\ldots$}{A[i j k $\ldots$]}
\bkey{element of A w/indices i, j, $\ldots$ in 1$^{st}$ dimension, k, l, $\ldots$ in second, $\ldots$}{A[i $\ldots$;  k $\ldots$; $\ldots$]}
\bkey{\rul}{}
\bkey{transpose of A}{$\invdiameter$A}
\bkey{transpose of B, axes ordered by A}{A$\invdiameter$B}
\bkey{maps A: 1 for a$\in$ B, 0 for a$\notin$ B}{A$\in$B}
\bkey{grade up A}{$\APLvert{\APLup}$A}
\bkey{grade up B with elements of A as top priority}{A$\APLvert{\APLup}$B}
\bkey{grade down A}{$\APLvert{\APLdown}$A}
\bkey{grade down B with elements of A as low priority}{A$\APLvert{\APLdown}$B}
\bkey{transpose of A}{$\invdiameter$A}
\bkey{enclose A}{$\subset$ A}
\bkey{enclose B with selected elements given the binary vector A}{A$\subset$ B}
\bkey{disclose A}{$\supset$ A}
\bkey{recursively pick elements of B given the indices in A}{A$\supset$ B}
\bkey{\rul}{}
\bkey{Decode single digits of B with respect to base A}{A$\bot$B}
\bkey{Encode B with respect to bases given by A}{A$\top$B}
\bkey{\rul}{}
\bkey{line label A}{A:}
\bkey{branch to line A}{$\rightarrow$A}
\bkey{\rul}{}
\bkey{execute APL expression A}{$\APLcirc{\bot}$A}
\bkey{format A as chars}{$\APLcirc{\top}$A}
\bkey{\rul}{}
\bkey{user input}{$\APLbox$}
\bkey{\rul}{}
\bkey{system var/function}{$\APLbox$}
\bkey{\rul}{}
\bkey{reduce {\tt op} over array A}{op/A}
\bkey{compress: select B using A as mask}{A/B}
\bkey{A/B on last dimension}{A$\notslash$B}
\bkey{expand: insert zeros in B using A as mask}{A$\backslash$B}
\bkey{A$\backslash$B on last dimension}{A$\notbackslash$B}
\bkey{inner product with functions f, g}{Af.gB}
\bkey{outer product with function f}{A$\circ$.fB}
\bkey{for each b$\in$B, apply: Ab}{A$\ddot{\;\;}$B}
\bkey{axis: AfC, over Bth axis}{Af[B]C}
\bkey{duplicate/commute}{}

\section{$\APLbox$CR, $\APLbox$FIO, $\APLbox$PLOT, $\APLbox$SQL}

When called with an empty string as right argument, these will show a table
with all their possible uses.

\section{Circle function}


% typesetting with tabs: TeX Book, p.232

\settabs\+\indent&$\APLminus$123 &Bsqrt($\APLminus$1+B$\times$B)......&$\APLminus$123  &D.......\cr
\+&{\tt A}& {\tt A}$\circ${\tt B}&{\tt A}&{\tt A}$\circ${\tt B}\cr
\+&&&&\cr
\+&  0 &$\sqrt{\hbox{\tt 1-B}\times\hbox{\tt B}}$     &    & \cr
\+& $\APLminus$1 & arcsin {\tt B}       &  1 & sin {\tt B} \cr
\+& $\APLminus$2 & arccos {\tt B}       &  2 & cos {\tt B} \cr
\+& $\APLminus$3 & arctan {\tt B}       &  3 & tan {\tt B} \cr
\+& $\APLminus$4 & $\sqrt{\APLminus\hbox{\tt 1+B}\times \hbox{\tt B}}$  &  4 & $\sqrt{\hbox{\tt 1+B} \times \hbox{\tt B}}$ \cr
\+& $\APLminus$5 & arcsinh {\tt B}      &  5 & sinh {\tt B} \cr
\+& $\APLminus$6 & arccosh {\tt B}      &  6 & cosh {\tt B} \cr
\+& $\APLminus$7 & arctanh {\tt B}      &  7 & tanh {\tt B} \cr
\+& $\APLminus$8 & $\APLminus${\tt (8}$\circ${\tt B)} &  8 & $\pm\sqrt{\APLminus\hbox{\tt 1+B}\times\hbox{\tt B}}$ \cr
%                         Z ←  sqrt($\APLminus$1+B×B) if B < 0
\+& $\APLminus$9 & {\tt B}              &  9 & real part of {\tt B} \cr
\+&$\APLminus$10 & {\tt +B}             & 10 & $\mid${\tt B} \cr
\+&$\APLminus$11 & {\tt 0J1}$\times${\tt B}   & 11 & imag part of {\tt B} \cr
\+&$\APLminus$12 & {\tt *0J1}$\times${\tt B} ($e^{iB}$) & 12 & arc {\tt B} (phase of {\tt B}) \cr

For {\tt A}$=8$, the sign before the square root is opposite of $\tt B$.

\section{Function Definition}

Example:
$f(d,v) = {\big(v_1^d+\cdots+v_n^d\big)^{1/d}}$

{\bf Dynamic function definition (dfn):}

$\alpha$ is the left argument, $\omega$ is the right argument.

{\tt f $\leftarrow$ \{ ( +/$\omega$*$\alpha$ ) * ($\div\alpha$)  \}}

{\bf Traditional function definition (tradfn):}

% $\APLdown$ marks begin/end of defun.
% ``{\tt $\APLdown$res $\leftarrow$ a f b ;u ;v}'' means
% ``function $f$ takes left argument {\tt a},
% right argument {\tt b}, uses local variables {\tt u}
% and {\tt v}, and returns its result in variable {\tt res}''.

% we can save one precious line making the above text a bit more terse:
$\APLdown$: begin/end defun.
``{\tt $\APLdown$R $\leftarrow$ A f B ;U ;V}'' is
``{\tt f} takes left arg {\tt A},
right arg {\tt B}, has local vars {\tt U},
{\tt V}, and returns result in {\tt R}''.

{\tt $\APLdown$res $\leftarrow$ d f v ;sq ;sum

$\;\;$  sq $\leftarrow$ v * d

$\;\;$  sum $\leftarrow$ +/sq

$\;\;$  res $\leftarrow$ sum*($\div$d)

$\APLdown$}

%\section{Error Recovery}
% do we really want this section?

%\section{Examples}

% some example showing control flow?
% commented? (will if fit?)

\copyrightnotice

\bye

% Local variables:
% compile-command: "pdftex aplcard"
% End:
